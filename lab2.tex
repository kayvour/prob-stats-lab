% Options for packages loaded elsewhere
\PassOptionsToPackage{unicode}{hyperref}
\PassOptionsToPackage{hyphens}{url}
%
\documentclass[
]{article}
\usepackage{amsmath,amssymb}
\usepackage{iftex}
\ifPDFTeX
  \usepackage[T1]{fontenc}
  \usepackage[utf8]{inputenc}
  \usepackage{textcomp} % provide euro and other symbols
\else % if luatex or xetex
  \usepackage{unicode-math} % this also loads fontspec
  \defaultfontfeatures{Scale=MatchLowercase}
  \defaultfontfeatures[\rmfamily]{Ligatures=TeX,Scale=1}
\fi
\usepackage{lmodern}
\ifPDFTeX\else
  % xetex/luatex font selection
\fi
% Use upquote if available, for straight quotes in verbatim environments
\IfFileExists{upquote.sty}{\usepackage{upquote}}{}
\IfFileExists{microtype.sty}{% use microtype if available
  \usepackage[]{microtype}
  \UseMicrotypeSet[protrusion]{basicmath} % disable protrusion for tt fonts
}{}
\makeatletter
\@ifundefined{KOMAClassName}{% if non-KOMA class
  \IfFileExists{parskip.sty}{%
    \usepackage{parskip}
  }{% else
    \setlength{\parindent}{0pt}
    \setlength{\parskip}{6pt plus 2pt minus 1pt}}
}{% if KOMA class
  \KOMAoptions{parskip=half}}
\makeatother
\usepackage{xcolor}
\usepackage[margin=1in]{geometry}
\usepackage{color}
\usepackage{fancyvrb}
\newcommand{\VerbBar}{|}
\newcommand{\VERB}{\Verb[commandchars=\\\{\}]}
\DefineVerbatimEnvironment{Highlighting}{Verbatim}{commandchars=\\\{\}}
% Add ',fontsize=\small' for more characters per line
\usepackage{framed}
\definecolor{shadecolor}{RGB}{248,248,248}
\newenvironment{Shaded}{\begin{snugshade}}{\end{snugshade}}
\newcommand{\AlertTok}[1]{\textcolor[rgb]{0.94,0.16,0.16}{#1}}
\newcommand{\AnnotationTok}[1]{\textcolor[rgb]{0.56,0.35,0.01}{\textbf{\textit{#1}}}}
\newcommand{\AttributeTok}[1]{\textcolor[rgb]{0.13,0.29,0.53}{#1}}
\newcommand{\BaseNTok}[1]{\textcolor[rgb]{0.00,0.00,0.81}{#1}}
\newcommand{\BuiltInTok}[1]{#1}
\newcommand{\CharTok}[1]{\textcolor[rgb]{0.31,0.60,0.02}{#1}}
\newcommand{\CommentTok}[1]{\textcolor[rgb]{0.56,0.35,0.01}{\textit{#1}}}
\newcommand{\CommentVarTok}[1]{\textcolor[rgb]{0.56,0.35,0.01}{\textbf{\textit{#1}}}}
\newcommand{\ConstantTok}[1]{\textcolor[rgb]{0.56,0.35,0.01}{#1}}
\newcommand{\ControlFlowTok}[1]{\textcolor[rgb]{0.13,0.29,0.53}{\textbf{#1}}}
\newcommand{\DataTypeTok}[1]{\textcolor[rgb]{0.13,0.29,0.53}{#1}}
\newcommand{\DecValTok}[1]{\textcolor[rgb]{0.00,0.00,0.81}{#1}}
\newcommand{\DocumentationTok}[1]{\textcolor[rgb]{0.56,0.35,0.01}{\textbf{\textit{#1}}}}
\newcommand{\ErrorTok}[1]{\textcolor[rgb]{0.64,0.00,0.00}{\textbf{#1}}}
\newcommand{\ExtensionTok}[1]{#1}
\newcommand{\FloatTok}[1]{\textcolor[rgb]{0.00,0.00,0.81}{#1}}
\newcommand{\FunctionTok}[1]{\textcolor[rgb]{0.13,0.29,0.53}{\textbf{#1}}}
\newcommand{\ImportTok}[1]{#1}
\newcommand{\InformationTok}[1]{\textcolor[rgb]{0.56,0.35,0.01}{\textbf{\textit{#1}}}}
\newcommand{\KeywordTok}[1]{\textcolor[rgb]{0.13,0.29,0.53}{\textbf{#1}}}
\newcommand{\NormalTok}[1]{#1}
\newcommand{\OperatorTok}[1]{\textcolor[rgb]{0.81,0.36,0.00}{\textbf{#1}}}
\newcommand{\OtherTok}[1]{\textcolor[rgb]{0.56,0.35,0.01}{#1}}
\newcommand{\PreprocessorTok}[1]{\textcolor[rgb]{0.56,0.35,0.01}{\textit{#1}}}
\newcommand{\RegionMarkerTok}[1]{#1}
\newcommand{\SpecialCharTok}[1]{\textcolor[rgb]{0.81,0.36,0.00}{\textbf{#1}}}
\newcommand{\SpecialStringTok}[1]{\textcolor[rgb]{0.31,0.60,0.02}{#1}}
\newcommand{\StringTok}[1]{\textcolor[rgb]{0.31,0.60,0.02}{#1}}
\newcommand{\VariableTok}[1]{\textcolor[rgb]{0.00,0.00,0.00}{#1}}
\newcommand{\VerbatimStringTok}[1]{\textcolor[rgb]{0.31,0.60,0.02}{#1}}
\newcommand{\WarningTok}[1]{\textcolor[rgb]{0.56,0.35,0.01}{\textbf{\textit{#1}}}}
\usepackage{graphicx}
\makeatletter
\newsavebox\pandoc@box
\newcommand*\pandocbounded[1]{% scales image to fit in text height/width
  \sbox\pandoc@box{#1}%
  \Gscale@div\@tempa{\textheight}{\dimexpr\ht\pandoc@box+\dp\pandoc@box\relax}%
  \Gscale@div\@tempb{\linewidth}{\wd\pandoc@box}%
  \ifdim\@tempb\p@<\@tempa\p@\let\@tempa\@tempb\fi% select the smaller of both
  \ifdim\@tempa\p@<\p@\scalebox{\@tempa}{\usebox\pandoc@box}%
  \else\usebox{\pandoc@box}%
  \fi%
}
% Set default figure placement to htbp
\def\fps@figure{htbp}
\makeatother
\setlength{\emergencystretch}{3em} % prevent overfull lines
\providecommand{\tightlist}{%
  \setlength{\itemsep}{0pt}\setlength{\parskip}{0pt}}
\setcounter{secnumdepth}{-\maxdimen} % remove section numbering
\usepackage{bookmark}
\IfFileExists{xurl.sty}{\usepackage{xurl}}{} % add URL line breaks if available
\urlstyle{same}
\hypersetup{
  pdftitle={lab2.R},
  pdfauthor={kaivalya},
  hidelinks,
  pdfcreator={LaTeX via pandoc}}

\title{lab2.R}
\author{kaivalya}
\date{2025-07-16}

\begin{document}
\maketitle

\begin{Shaded}
\begin{Highlighting}[]
\FunctionTok{R.home}\NormalTok{(}\StringTok{"bin"}\NormalTok{)}
\end{Highlighting}
\end{Shaded}

\begin{verbatim}
## [1] "C:/PROGRA~1/R/R-45~1.1/bin/x64"
\end{verbatim}

\begin{Shaded}
\begin{Highlighting}[]
\DecValTok{1} \SpecialCharTok{:} \DecValTok{10}
\end{Highlighting}
\end{Shaded}

\begin{verbatim}
##  [1]  1  2  3  4  5  6  7  8  9 10
\end{verbatim}

\begin{Shaded}
\begin{Highlighting}[]
\FunctionTok{seq}\NormalTok{(}\DecValTok{1}\NormalTok{, }\DecValTok{10}\NormalTok{, }\AttributeTok{by =} \DecValTok{2}\NormalTok{)}
\end{Highlighting}
\end{Shaded}

\begin{verbatim}
## [1] 1 3 5 7 9
\end{verbatim}

\begin{Shaded}
\begin{Highlighting}[]
\NormalTok{x }\OtherTok{=} \FunctionTok{c}\NormalTok{(}\DecValTok{1} \SpecialCharTok{:} \DecValTok{10}\NormalTok{)}
\FunctionTok{sum}\NormalTok{(x)}
\end{Highlighting}
\end{Shaded}

\begin{verbatim}
## [1] 55
\end{verbatim}

\begin{Shaded}
\begin{Highlighting}[]
\FunctionTok{mean}\NormalTok{(x)}
\end{Highlighting}
\end{Shaded}

\begin{verbatim}
## [1] 5.5
\end{verbatim}

\begin{Shaded}
\begin{Highlighting}[]
\NormalTok{y }\OtherTok{=} \FunctionTok{c}\NormalTok{(}\DecValTok{1}\NormalTok{, }\DecValTok{2}\NormalTok{, }\ConstantTok{NA}\NormalTok{)}
\FunctionTok{sum}\NormalTok{(y, }\AttributeTok{na.rm =} \ConstantTok{TRUE}\NormalTok{)}
\end{Highlighting}
\end{Shaded}

\begin{verbatim}
## [1] 3
\end{verbatim}

\begin{Shaded}
\begin{Highlighting}[]
\FunctionTok{rep}\NormalTok{(}\DecValTok{5}\NormalTok{, }\AttributeTok{times =} \DecValTok{3}\NormalTok{)}
\end{Highlighting}
\end{Shaded}

\begin{verbatim}
## [1] 5 5 5
\end{verbatim}

\begin{Shaded}
\begin{Highlighting}[]
\FunctionTok{rep}\NormalTok{(}\DecValTok{1} \SpecialCharTok{:} \DecValTok{10}\NormalTok{, }\AttributeTok{times =} \DecValTok{2}\NormalTok{)}
\end{Highlighting}
\end{Shaded}

\begin{verbatim}
##  [1]  1  2  3  4  5  6  7  8  9 10  1  2  3  4  5  6  7  8  9 10
\end{verbatim}

\begin{Shaded}
\begin{Highlighting}[]
\CommentTok{\#generate table}
\CommentTok{\#data.frame(var1, var2, ...)}
\NormalTok{EmpId }\OtherTok{=} \FunctionTok{c}\NormalTok{(}\DecValTok{101}\NormalTok{, }\DecValTok{102}\NormalTok{, }\DecValTok{103}\NormalTok{, }\DecValTok{104}\NormalTok{)}
\NormalTok{Name }\OtherTok{=} \FunctionTok{c}\NormalTok{(}\StringTok{"x"}\NormalTok{, }\StringTok{"y"}\NormalTok{, }\StringTok{"z"}\NormalTok{, }\StringTok{"w"}\NormalTok{)}
\NormalTok{Age }\OtherTok{=} \FunctionTok{c}\NormalTok{(}\DecValTok{30}\NormalTok{, }\DecValTok{40}\NormalTok{, }\DecValTok{50}\NormalTok{, }\DecValTok{55}\NormalTok{)}
\NormalTok{Department }\OtherTok{=} \FunctionTok{c}\NormalTok{(}\StringTok{"Mathematics"}\NormalTok{, }\StringTok{"Physics"}\NormalTok{, }\StringTok{"Chemistry"}\NormalTok{, }\StringTok{"CS"}\NormalTok{)}
\NormalTok{Salary }\OtherTok{=} \FunctionTok{c}\NormalTok{(}\DecValTok{80000}\NormalTok{, }\DecValTok{85000}\NormalTok{, }\DecValTok{50000}\NormalTok{, }\DecValTok{55000}\NormalTok{)}

\NormalTok{Employee\_data }\OtherTok{=} \FunctionTok{data.frame}\NormalTok{(EmpId, Name, Age, Department, Salary)}
\FunctionTok{print}\NormalTok{(Employee\_data)}
\end{Highlighting}
\end{Shaded}

\begin{verbatim}
##   EmpId Name Age  Department Salary
## 1   101    x  30 Mathematics  80000
## 2   102    y  40     Physics  85000
## 3   103    z  50   Chemistry  50000
## 4   104    w  55          CS  55000
\end{verbatim}

\begin{Shaded}
\begin{Highlighting}[]
\NormalTok{Employee\_data[}\DecValTok{1}\NormalTok{, ]}
\end{Highlighting}
\end{Shaded}

\begin{verbatim}
##   EmpId Name Age  Department Salary
## 1   101    x  30 Mathematics  80000
\end{verbatim}

\begin{Shaded}
\begin{Highlighting}[]
\NormalTok{Employee\_data[ , }\DecValTok{3}\NormalTok{]}
\end{Highlighting}
\end{Shaded}

\begin{verbatim}
## [1] 30 40 50 55
\end{verbatim}

\begin{Shaded}
\begin{Highlighting}[]
\NormalTok{Employee\_data[}\DecValTok{1}\NormalTok{, }\DecValTok{2}\NormalTok{]}
\end{Highlighting}
\end{Shaded}

\begin{verbatim}
## [1] "x"
\end{verbatim}

\begin{Shaded}
\begin{Highlighting}[]
\NormalTok{Employee\_data }\SpecialCharTok{$}\NormalTok{ Bonus }\OtherTok{=} \FunctionTok{c}\NormalTok{(}\DecValTok{20000}\NormalTok{, }\DecValTok{30000}\NormalTok{, }\DecValTok{10000}\NormalTok{, }\DecValTok{15000}\NormalTok{)}
\FunctionTok{print}\NormalTok{(Employee\_data)}
\end{Highlighting}
\end{Shaded}

\begin{verbatim}
##   EmpId Name Age  Department Salary Bonus
## 1   101    x  30 Mathematics  80000 20000
## 2   102    y  40     Physics  85000 30000
## 3   103    z  50   Chemistry  50000 10000
## 4   104    w  55          CS  55000 15000
\end{verbatim}

\begin{Shaded}
\begin{Highlighting}[]
\CommentTok{\#Q.1 Create a table with the following attributes:}
\CommentTok{\#Name, RollNo, Age and Marks for 10 students}
\CommentTok{\#Extract the name column from the table}
\CommentTok{\#Find the averages of the marks}

\CommentTok{\#Q.2 Create a table of 5 products with ProductID, ProductName, Price and Quantity}
\CommentTok{\#Calculate Total Cost = Price x Quantity and add it as a new column}

\NormalTok{StudName }\OtherTok{=} \FunctionTok{c}\NormalTok{(}\StringTok{"Aarav"}\NormalTok{, }\StringTok{"Diya"}\NormalTok{, }\StringTok{"Rohan"}\NormalTok{, }\StringTok{"Ishita"}\NormalTok{, }\StringTok{"Kabir"}\NormalTok{, }\StringTok{"Anaya"}\NormalTok{, }\StringTok{"Vivaan"}\NormalTok{, }\StringTok{"Meera"}\NormalTok{, }\StringTok{"Aditya"}\NormalTok{, }\StringTok{"Sneha"}\NormalTok{)}
\NormalTok{RollNo }\OtherTok{=} \FunctionTok{c}\NormalTok{(}\DecValTok{1}\NormalTok{, }\DecValTok{2}\NormalTok{, }\DecValTok{3}\NormalTok{, }\DecValTok{4}\NormalTok{, }\DecValTok{5}\NormalTok{, }\DecValTok{6}\NormalTok{, }\DecValTok{7}\NormalTok{, }\DecValTok{8}\NormalTok{, }\DecValTok{9}\NormalTok{, }\DecValTok{10}\NormalTok{)}
\NormalTok{Age }\OtherTok{=} \FunctionTok{c}\NormalTok{(}\DecValTok{17}\NormalTok{, }\DecValTok{16}\NormalTok{, }\DecValTok{17}\NormalTok{, }\DecValTok{17}\NormalTok{, }\DecValTok{18}\NormalTok{, }\DecValTok{17}\NormalTok{, }\DecValTok{18}\NormalTok{, }\DecValTok{16}\NormalTok{, }\DecValTok{17}\NormalTok{, }\DecValTok{18}\NormalTok{)}
\NormalTok{Marks }\OtherTok{=} \FunctionTok{c}\NormalTok{(}\DecValTok{85}\NormalTok{, }\DecValTok{92}\NormalTok{, }\DecValTok{78}\NormalTok{, }\DecValTok{88}\NormalTok{, }\DecValTok{69}\NormalTok{, }\DecValTok{95}\NormalTok{, }\DecValTok{73}\NormalTok{, }\DecValTok{81}\NormalTok{, }\DecValTok{90}\NormalTok{, }\DecValTok{87}\NormalTok{)}
\NormalTok{Student\_data }\OtherTok{=} \FunctionTok{data.frame}\NormalTok{(StudName, RollNo, Age, Marks)}

\FunctionTok{print}\NormalTok{(Student\_data }\SpecialCharTok{$}\NormalTok{ StudName)}
\end{Highlighting}
\end{Shaded}

\begin{verbatim}
##  [1] "Aarav"  "Diya"   "Rohan"  "Ishita" "Kabir"  "Anaya"  "Vivaan" "Meera" 
##  [9] "Aditya" "Sneha"
\end{verbatim}

\begin{Shaded}
\begin{Highlighting}[]
\FunctionTok{print}\NormalTok{(}\FunctionTok{mean}\NormalTok{(Student\_data }\SpecialCharTok{$}\NormalTok{ Marks))}
\end{Highlighting}
\end{Shaded}

\begin{verbatim}
## [1] 83.8
\end{verbatim}

\begin{Shaded}
\begin{Highlighting}[]
\NormalTok{ProductId }\OtherTok{=} \FunctionTok{c}\NormalTok{(}\StringTok{"P101"}\NormalTok{, }\StringTok{"P102"}\NormalTok{, }\StringTok{"P103"}\NormalTok{, }\StringTok{"P104"}\NormalTok{, }\StringTok{"P105"}\NormalTok{)}
\NormalTok{ProductName }\OtherTok{=} \FunctionTok{c}\NormalTok{(}\StringTok{"Laptop"}\NormalTok{, }\StringTok{"Smartphone"}\NormalTok{, }\StringTok{"Headphones"}\NormalTok{, }\StringTok{"Keyboard"}\NormalTok{, }\StringTok{"Monitor"}\NormalTok{)}
\NormalTok{Price }\OtherTok{=} \FunctionTok{c}\NormalTok{(}\DecValTok{60000}\NormalTok{, }\DecValTok{25000}\NormalTok{, }\DecValTok{2000}\NormalTok{, }\DecValTok{1500}\NormalTok{, }\DecValTok{12000}\NormalTok{)}
\NormalTok{Quantity }\OtherTok{=} \FunctionTok{c}\NormalTok{(}\DecValTok{10}\NormalTok{, }\DecValTok{25}\NormalTok{, }\DecValTok{50}\NormalTok{, }\DecValTok{40}\NormalTok{, }\DecValTok{20}\NormalTok{)}
\NormalTok{Product\_data }\OtherTok{=} \FunctionTok{data.frame}\NormalTok{(ProductId, ProductName, Price, Quantity)}

\NormalTok{Product\_data }\SpecialCharTok{$}\NormalTok{ TotalCost }\OtherTok{=}\NormalTok{ Product\_data }\SpecialCharTok{$}\NormalTok{ Price }\SpecialCharTok{*}\NormalTok{ Product\_data }\SpecialCharTok{$}\NormalTok{ Quantity}
\FunctionTok{print}\NormalTok{(Product\_data)}
\end{Highlighting}
\end{Shaded}

\begin{verbatim}
##   ProductId ProductName Price Quantity TotalCost
## 1      P101      Laptop 60000       10    600000
## 2      P102  Smartphone 25000       25    625000
## 3      P103  Headphones  2000       50    100000
## 4      P104    Keyboard  1500       40     60000
## 5      P105     Monitor 12000       20    240000
\end{verbatim}

\begin{Shaded}
\begin{Highlighting}[]
\CommentTok{\#matrix}
\NormalTok{M }\OtherTok{=} \FunctionTok{matrix}\NormalTok{(}\DecValTok{1} \SpecialCharTok{:} \DecValTok{9}\NormalTok{, }\AttributeTok{nrow =} \DecValTok{3}\NormalTok{, }\AttributeTok{ncol =} \DecValTok{3}\NormalTok{, }\AttributeTok{byrow =} \ConstantTok{TRUE}\NormalTok{)}
\FunctionTok{print}\NormalTok{(M)}
\end{Highlighting}
\end{Shaded}

\begin{verbatim}
##      [,1] [,2] [,3]
## [1,]    1    2    3
## [2,]    4    5    6
## [3,]    7    8    9
\end{verbatim}

\begin{Shaded}
\begin{Highlighting}[]
\CommentTok{\#packages}
\FunctionTok{library}\NormalTok{(readxl)}
\NormalTok{Book1 }\OtherTok{\textless{}{-}} \FunctionTok{read\_excel}\NormalTok{(}\StringTok{"C:/Users/vaidy/Documents/socrates/Philosophy\_Observances.xlsx"}\NormalTok{)}
\FunctionTok{View}\NormalTok{(Book1)}

\CommentTok{\#data}
\FunctionTok{data}\NormalTok{(mtcars)}
\FunctionTok{View}\NormalTok{(mtcars)}
\end{Highlighting}
\end{Shaded}


\end{document}
